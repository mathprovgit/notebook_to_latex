\documentclass[reprint, floatfix, groupaddress, prb]{article}

%set main font
\usepackage[T1]{fontenc}
\usepackage{lmodern}

%depth of toc
\setcounter{tocdepth}{3}

\AtBeginDocument{
\heavyrulewidth=.08em
\lightrulewidth=.05em
\cmidrulewidth=.03em
\belowbottomsep=0pt
\aboverulesep=.4ex
\abovetopsep=0pt
\cmidrulesep=\doublerulesep
\cmidrulekern=.5em
\defaultaddspace=.5em
}

\usepackage{array}
\usepackage{placeins}
\usepackage{graphicx}
\usepackage{multirow}
\usepackage[table]{xcolor}

% new page before each chapter
\usepackage{sectsty}
\sectionfont{\clearpage}



    \usepackage[breakable]{tcolorbox}
    \usepackage{parskip} % Stop auto-indenting (to mimic markdown behaviour)
    
    \usepackage{iftex}
    \ifPDFTeX
    	\usepackage[T1]{fontenc}
    	\usepackage{mathpazo}
    \else
    	\usepackage{fontspec}
    \fi

    % Basic figure setup, for now with no caption control since it's done
    % automatically by Pandoc (which extracts ![](path) syntax from Markdown).
    \usepackage{graphicx}
    % Maintain compatibility with old templates. Remove in nbconvert 6.0
    \let\Oldincludegraphics\includegraphics
    % Ensure that by default, figures have no caption (until we provide a
    % proper Figure object with a Caption API and a way to capture that
    % in the conversion process - todo).
    \usepackage{caption}
    \DeclareCaptionFormat{nocaption}{}
    \captionsetup{format=nocaption,aboveskip=0pt,belowskip=0pt}

    \usepackage[Export]{adjustbox} % Used to constrain images to a maximum size
    \adjustboxset{max size={0.9\linewidth}{0.9\paperheight}}
    \usepackage{float}
    \floatplacement{figure}{H} % forces figures to be placed at the correct location
    \usepackage{xcolor} % Allow colors to be defined
    \usepackage{enumerate} % Needed for markdown enumerations to work
    \usepackage{geometry} % Used to adjust the document margins
    \usepackage{amsmath} % Equations
    \usepackage{amssymb} % Equations
    \usepackage{textcomp} % defines textquotesingle
    % Hack from http://tex.stackexchange.com/a/47451/13684:
    \AtBeginDocument{%
        \def\PYZsq{\textquotesingle}% Upright quotes in Pygmentized code
    }
    \usepackage{upquote} % Upright quotes for verbatim code
    \usepackage{eurosym} % defines \euro
    \usepackage[mathletters]{ucs} % Extended unicode (utf-8) support
    \usepackage{fancyvrb} % verbatim replacement that allows latex
    \usepackage{grffile} % extends the file name processing of package graphics 
                         % to support a larger range
    \makeatletter % fix for grffile with XeLaTeX
    \def\Gread@@xetex#1{%
      \IfFileExists{"\Gin@base".bb}%
      {\Gread@eps{\Gin@base.bb}}%
      {\Gread@@xetex@aux#1}%
    }
    \makeatother

    % The hyperref package gives us a pdf with properly built
    % internal navigation ('pdf bookmarks' for the table of contents,
    % internal cross-reference links, web links for URLs, etc.)
    \usepackage{hyperref}
    % The default LaTeX title has an obnoxious amount of whitespace. By default,
    % titling removes some of it. It also provides customization options.
    \usepackage{titling}
    \usepackage{longtable} % longtable support required by pandoc >1.10
    \usepackage{booktabs}  % table support for pandoc > 1.12.2
    \usepackage[inline]{enumitem} % IRkernel/repr support (it uses the enumerate* environment)
    \usepackage[normalem]{ulem} % ulem is needed to support strikethroughs (\sout)
                                % normalem makes italics be italics, not underlines
    \usepackage{mathrsfs}
    

    
    % Colors for the hyperref package
    \definecolor{urlcolor}{rgb}{0,.145,.698}
    \definecolor{linkcolor}{rgb}{.71,0.21,0.01}
    \definecolor{citecolor}{rgb}{.12,.54,.11}

    % ANSI colors
    \definecolor{ansi-black}{HTML}{3E424D}
    \definecolor{ansi-black-intense}{HTML}{282C36}
    \definecolor{ansi-red}{HTML}{E75C58}
    \definecolor{ansi-red-intense}{HTML}{B22B31}
    \definecolor{ansi-green}{HTML}{00A250}
    \definecolor{ansi-green-intense}{HTML}{007427}
    \definecolor{ansi-yellow}{HTML}{DDB62B}
    \definecolor{ansi-yellow-intense}{HTML}{B27D12}
    \definecolor{ansi-blue}{HTML}{208FFB}
    \definecolor{ansi-blue-intense}{HTML}{0065CA}
    \definecolor{ansi-magenta}{HTML}{D160C4}
    \definecolor{ansi-magenta-intense}{HTML}{A03196}
    \definecolor{ansi-cyan}{HTML}{60C6C8}
    \definecolor{ansi-cyan-intense}{HTML}{258F8F}
    \definecolor{ansi-white}{HTML}{C5C1B4}
    \definecolor{ansi-white-intense}{HTML}{A1A6B2}
    \definecolor{ansi-default-inverse-fg}{HTML}{FFFFFF}
    \definecolor{ansi-default-inverse-bg}{HTML}{000000}

    % commands and environments needed by pandoc snippets
    % extracted from the output of `pandoc -s`
    \providecommand{\tightlist}{%
      \setlength{\itemsep}{0pt}\setlength{\parskip}{0pt}}
    \DefineVerbatimEnvironment{Highlighting}{Verbatim}{commandchars=\\\{\}}
    % Add ',fontsize=\small' for more characters per line
    \newenvironment{Shaded}{}{}
    \newcommand{\KeywordTok}[1]{\textcolor[rgb]{0.00,0.44,0.13}{\textbf{{#1}}}}
    \newcommand{\DataTypeTok}[1]{\textcolor[rgb]{0.56,0.13,0.00}{{#1}}}
    \newcommand{\DecValTok}[1]{\textcolor[rgb]{0.25,0.63,0.44}{{#1}}}
    \newcommand{\BaseNTok}[1]{\textcolor[rgb]{0.25,0.63,0.44}{{#1}}}
    \newcommand{\FloatTok}[1]{\textcolor[rgb]{0.25,0.63,0.44}{{#1}}}
    \newcommand{\CharTok}[1]{\textcolor[rgb]{0.25,0.44,0.63}{{#1}}}
    \newcommand{\StringTok}[1]{\textcolor[rgb]{0.25,0.44,0.63}{{#1}}}
    \newcommand{\CommentTok}[1]{\textcolor[rgb]{0.38,0.63,0.69}{\textit{{#1}}}}
    \newcommand{\OtherTok}[1]{\textcolor[rgb]{0.00,0.44,0.13}{{#1}}}
    \newcommand{\AlertTok}[1]{\textcolor[rgb]{1.00,0.00,0.00}{\textbf{{#1}}}}
    \newcommand{\FunctionTok}[1]{\textcolor[rgb]{0.02,0.16,0.49}{{#1}}}
    \newcommand{\RegionMarkerTok}[1]{{#1}}
    \newcommand{\ErrorTok}[1]{\textcolor[rgb]{1.00,0.00,0.00}{\textbf{{#1}}}}
    \newcommand{\NormalTok}[1]{{#1}}
    
    % Additional commands for more recent versions of Pandoc
    \newcommand{\ConstantTok}[1]{\textcolor[rgb]{0.53,0.00,0.00}{{#1}}}
    \newcommand{\SpecialCharTok}[1]{\textcolor[rgb]{0.25,0.44,0.63}{{#1}}}
    \newcommand{\VerbatimStringTok}[1]{\textcolor[rgb]{0.25,0.44,0.63}{{#1}}}
    \newcommand{\SpecialStringTok}[1]{\textcolor[rgb]{0.73,0.40,0.53}{{#1}}}
    \newcommand{\ImportTok}[1]{{#1}}
    \newcommand{\DocumentationTok}[1]{\textcolor[rgb]{0.73,0.13,0.13}{\textit{{#1}}}}
    \newcommand{\AnnotationTok}[1]{\textcolor[rgb]{0.38,0.63,0.69}{\textbf{\textit{{#1}}}}}
    \newcommand{\CommentVarTok}[1]{\textcolor[rgb]{0.38,0.63,0.69}{\textbf{\textit{{#1}}}}}
    \newcommand{\VariableTok}[1]{\textcolor[rgb]{0.10,0.09,0.49}{{#1}}}
    \newcommand{\ControlFlowTok}[1]{\textcolor[rgb]{0.00,0.44,0.13}{\textbf{{#1}}}}
    \newcommand{\OperatorTok}[1]{\textcolor[rgb]{0.40,0.40,0.40}{{#1}}}
    \newcommand{\BuiltInTok}[1]{{#1}}
    \newcommand{\ExtensionTok}[1]{{#1}}
    \newcommand{\PreprocessorTok}[1]{\textcolor[rgb]{0.74,0.48,0.00}{{#1}}}
    \newcommand{\AttributeTok}[1]{\textcolor[rgb]{0.49,0.56,0.16}{{#1}}}
    \newcommand{\InformationTok}[1]{\textcolor[rgb]{0.38,0.63,0.69}{\textbf{\textit{{#1}}}}}
    \newcommand{\WarningTok}[1]{\textcolor[rgb]{0.38,0.63,0.69}{\textbf{\textit{{#1}}}}}
    
    
    % Define a nice break command that doesn't care if a line doesn't already
    % exist.
    \def\br{\hspace*{\fill} \\* }
    % Math Jax compatibility definitions
    \def\gt{>}
    \def\lt{<}
    \let\Oldtex\TeX
    \let\Oldlatex\LaTeX
    \renewcommand{\TeX}{\textrm{\Oldtex}}
    \renewcommand{\LaTeX}{\textrm{\Oldlatex}}
    % Document parameters
    % Document title
    

\title{\huge{\textbf{My Latex automated Report}}\\[2ex]  \LARGE{Generated from my notebook.ipynb}}
\author{Mathieu Provost}
\date{\today}
   
\makeatletter         
\def\@maketitle{
\raggedright

\begin{center}
\includegraphics[width=30mm]{pics/jupyter_logo.png}\\[4ex]
{\@title }\\[6ex] 
{\Large  \@author}\\[4ex] 
\@date\\[8ex]
\includegraphics{pics/jupyter_picture.jpg}
\end{center}}

% newline after paragraph
\makeatletter
\renewcommand\paragraph{%
    \@startsection{paragraph}{4}{0mm}%
       {-\baselineskip}%
       {.5\baselineskip}%
       {\normalfont\normalsize\bfseries}}
\setcounter{secnumdepth}{4}
\makeatother


    
    
    
    
    
% Pygments definitions
\makeatletter
\def\PY@reset{\let\PY@it=\relax \let\PY@bf=\relax%
    \let\PY@ul=\relax \let\PY@tc=\relax%
    \let\PY@bc=\relax \let\PY@ff=\relax}
\def\PY@tok#1{\csname PY@tok@#1\endcsname}
\def\PY@toks#1+{\ifx\relax#1\empty\else%
    \PY@tok{#1}\expandafter\PY@toks\fi}
\def\PY@do#1{\PY@bc{\PY@tc{\PY@ul{%
    \PY@it{\PY@bf{\PY@ff{#1}}}}}}}
\def\PY#1#2{\PY@reset\PY@toks#1+\relax+\PY@do{#2}}

\expandafter\def\csname PY@tok@w\endcsname{\def\PY@tc##1{\textcolor[rgb]{0.73,0.73,0.73}{##1}}}
\expandafter\def\csname PY@tok@c\endcsname{\let\PY@it=\textit\def\PY@tc##1{\textcolor[rgb]{0.25,0.50,0.50}{##1}}}
\expandafter\def\csname PY@tok@cp\endcsname{\def\PY@tc##1{\textcolor[rgb]{0.74,0.48,0.00}{##1}}}
\expandafter\def\csname PY@tok@k\endcsname{\let\PY@bf=\textbf\def\PY@tc##1{\textcolor[rgb]{0.00,0.50,0.00}{##1}}}
\expandafter\def\csname PY@tok@kp\endcsname{\def\PY@tc##1{\textcolor[rgb]{0.00,0.50,0.00}{##1}}}
\expandafter\def\csname PY@tok@kt\endcsname{\def\PY@tc##1{\textcolor[rgb]{0.69,0.00,0.25}{##1}}}
\expandafter\def\csname PY@tok@o\endcsname{\def\PY@tc##1{\textcolor[rgb]{0.40,0.40,0.40}{##1}}}
\expandafter\def\csname PY@tok@ow\endcsname{\let\PY@bf=\textbf\def\PY@tc##1{\textcolor[rgb]{0.67,0.13,1.00}{##1}}}
\expandafter\def\csname PY@tok@nb\endcsname{\def\PY@tc##1{\textcolor[rgb]{0.00,0.50,0.00}{##1}}}
\expandafter\def\csname PY@tok@nf\endcsname{\def\PY@tc##1{\textcolor[rgb]{0.00,0.00,1.00}{##1}}}
\expandafter\def\csname PY@tok@nc\endcsname{\let\PY@bf=\textbf\def\PY@tc##1{\textcolor[rgb]{0.00,0.00,1.00}{##1}}}
\expandafter\def\csname PY@tok@nn\endcsname{\let\PY@bf=\textbf\def\PY@tc##1{\textcolor[rgb]{0.00,0.00,1.00}{##1}}}
\expandafter\def\csname PY@tok@ne\endcsname{\let\PY@bf=\textbf\def\PY@tc##1{\textcolor[rgb]{0.82,0.25,0.23}{##1}}}
\expandafter\def\csname PY@tok@nv\endcsname{\def\PY@tc##1{\textcolor[rgb]{0.10,0.09,0.49}{##1}}}
\expandafter\def\csname PY@tok@no\endcsname{\def\PY@tc##1{\textcolor[rgb]{0.53,0.00,0.00}{##1}}}
\expandafter\def\csname PY@tok@nl\endcsname{\def\PY@tc##1{\textcolor[rgb]{0.63,0.63,0.00}{##1}}}
\expandafter\def\csname PY@tok@ni\endcsname{\let\PY@bf=\textbf\def\PY@tc##1{\textcolor[rgb]{0.60,0.60,0.60}{##1}}}
\expandafter\def\csname PY@tok@na\endcsname{\def\PY@tc##1{\textcolor[rgb]{0.49,0.56,0.16}{##1}}}
\expandafter\def\csname PY@tok@nt\endcsname{\let\PY@bf=\textbf\def\PY@tc##1{\textcolor[rgb]{0.00,0.50,0.00}{##1}}}
\expandafter\def\csname PY@tok@nd\endcsname{\def\PY@tc##1{\textcolor[rgb]{0.67,0.13,1.00}{##1}}}
\expandafter\def\csname PY@tok@s\endcsname{\def\PY@tc##1{\textcolor[rgb]{0.73,0.13,0.13}{##1}}}
\expandafter\def\csname PY@tok@sd\endcsname{\let\PY@it=\textit\def\PY@tc##1{\textcolor[rgb]{0.73,0.13,0.13}{##1}}}
\expandafter\def\csname PY@tok@si\endcsname{\let\PY@bf=\textbf\def\PY@tc##1{\textcolor[rgb]{0.73,0.40,0.53}{##1}}}
\expandafter\def\csname PY@tok@se\endcsname{\let\PY@bf=\textbf\def\PY@tc##1{\textcolor[rgb]{0.73,0.40,0.13}{##1}}}
\expandafter\def\csname PY@tok@sr\endcsname{\def\PY@tc##1{\textcolor[rgb]{0.73,0.40,0.53}{##1}}}
\expandafter\def\csname PY@tok@ss\endcsname{\def\PY@tc##1{\textcolor[rgb]{0.10,0.09,0.49}{##1}}}
\expandafter\def\csname PY@tok@sx\endcsname{\def\PY@tc##1{\textcolor[rgb]{0.00,0.50,0.00}{##1}}}
\expandafter\def\csname PY@tok@m\endcsname{\def\PY@tc##1{\textcolor[rgb]{0.40,0.40,0.40}{##1}}}
\expandafter\def\csname PY@tok@gh\endcsname{\let\PY@bf=\textbf\def\PY@tc##1{\textcolor[rgb]{0.00,0.00,0.50}{##1}}}
\expandafter\def\csname PY@tok@gu\endcsname{\let\PY@bf=\textbf\def\PY@tc##1{\textcolor[rgb]{0.50,0.00,0.50}{##1}}}
\expandafter\def\csname PY@tok@gd\endcsname{\def\PY@tc##1{\textcolor[rgb]{0.63,0.00,0.00}{##1}}}
\expandafter\def\csname PY@tok@gi\endcsname{\def\PY@tc##1{\textcolor[rgb]{0.00,0.63,0.00}{##1}}}
\expandafter\def\csname PY@tok@gr\endcsname{\def\PY@tc##1{\textcolor[rgb]{1.00,0.00,0.00}{##1}}}
\expandafter\def\csname PY@tok@ge\endcsname{\let\PY@it=\textit}
\expandafter\def\csname PY@tok@gs\endcsname{\let\PY@bf=\textbf}
\expandafter\def\csname PY@tok@gp\endcsname{\let\PY@bf=\textbf\def\PY@tc##1{\textcolor[rgb]{0.00,0.00,0.50}{##1}}}
\expandafter\def\csname PY@tok@go\endcsname{\def\PY@tc##1{\textcolor[rgb]{0.53,0.53,0.53}{##1}}}
\expandafter\def\csname PY@tok@gt\endcsname{\def\PY@tc##1{\textcolor[rgb]{0.00,0.27,0.87}{##1}}}
\expandafter\def\csname PY@tok@err\endcsname{\def\PY@bc##1{\setlength{\fboxsep}{0pt}\fcolorbox[rgb]{1.00,0.00,0.00}{1,1,1}{\strut ##1}}}
\expandafter\def\csname PY@tok@kc\endcsname{\let\PY@bf=\textbf\def\PY@tc##1{\textcolor[rgb]{0.00,0.50,0.00}{##1}}}
\expandafter\def\csname PY@tok@kd\endcsname{\let\PY@bf=\textbf\def\PY@tc##1{\textcolor[rgb]{0.00,0.50,0.00}{##1}}}
\expandafter\def\csname PY@tok@kn\endcsname{\let\PY@bf=\textbf\def\PY@tc##1{\textcolor[rgb]{0.00,0.50,0.00}{##1}}}
\expandafter\def\csname PY@tok@kr\endcsname{\let\PY@bf=\textbf\def\PY@tc##1{\textcolor[rgb]{0.00,0.50,0.00}{##1}}}
\expandafter\def\csname PY@tok@bp\endcsname{\def\PY@tc##1{\textcolor[rgb]{0.00,0.50,0.00}{##1}}}
\expandafter\def\csname PY@tok@fm\endcsname{\def\PY@tc##1{\textcolor[rgb]{0.00,0.00,1.00}{##1}}}
\expandafter\def\csname PY@tok@vc\endcsname{\def\PY@tc##1{\textcolor[rgb]{0.10,0.09,0.49}{##1}}}
\expandafter\def\csname PY@tok@vg\endcsname{\def\PY@tc##1{\textcolor[rgb]{0.10,0.09,0.49}{##1}}}
\expandafter\def\csname PY@tok@vi\endcsname{\def\PY@tc##1{\textcolor[rgb]{0.10,0.09,0.49}{##1}}}
\expandafter\def\csname PY@tok@vm\endcsname{\def\PY@tc##1{\textcolor[rgb]{0.10,0.09,0.49}{##1}}}
\expandafter\def\csname PY@tok@sa\endcsname{\def\PY@tc##1{\textcolor[rgb]{0.73,0.13,0.13}{##1}}}
\expandafter\def\csname PY@tok@sb\endcsname{\def\PY@tc##1{\textcolor[rgb]{0.73,0.13,0.13}{##1}}}
\expandafter\def\csname PY@tok@sc\endcsname{\def\PY@tc##1{\textcolor[rgb]{0.73,0.13,0.13}{##1}}}
\expandafter\def\csname PY@tok@dl\endcsname{\def\PY@tc##1{\textcolor[rgb]{0.73,0.13,0.13}{##1}}}
\expandafter\def\csname PY@tok@s2\endcsname{\def\PY@tc##1{\textcolor[rgb]{0.73,0.13,0.13}{##1}}}
\expandafter\def\csname PY@tok@sh\endcsname{\def\PY@tc##1{\textcolor[rgb]{0.73,0.13,0.13}{##1}}}
\expandafter\def\csname PY@tok@s1\endcsname{\def\PY@tc##1{\textcolor[rgb]{0.73,0.13,0.13}{##1}}}
\expandafter\def\csname PY@tok@mb\endcsname{\def\PY@tc##1{\textcolor[rgb]{0.40,0.40,0.40}{##1}}}
\expandafter\def\csname PY@tok@mf\endcsname{\def\PY@tc##1{\textcolor[rgb]{0.40,0.40,0.40}{##1}}}
\expandafter\def\csname PY@tok@mh\endcsname{\def\PY@tc##1{\textcolor[rgb]{0.40,0.40,0.40}{##1}}}
\expandafter\def\csname PY@tok@mi\endcsname{\def\PY@tc##1{\textcolor[rgb]{0.40,0.40,0.40}{##1}}}
\expandafter\def\csname PY@tok@il\endcsname{\def\PY@tc##1{\textcolor[rgb]{0.40,0.40,0.40}{##1}}}
\expandafter\def\csname PY@tok@mo\endcsname{\def\PY@tc##1{\textcolor[rgb]{0.40,0.40,0.40}{##1}}}
\expandafter\def\csname PY@tok@ch\endcsname{\let\PY@it=\textit\def\PY@tc##1{\textcolor[rgb]{0.25,0.50,0.50}{##1}}}
\expandafter\def\csname PY@tok@cm\endcsname{\let\PY@it=\textit\def\PY@tc##1{\textcolor[rgb]{0.25,0.50,0.50}{##1}}}
\expandafter\def\csname PY@tok@cpf\endcsname{\let\PY@it=\textit\def\PY@tc##1{\textcolor[rgb]{0.25,0.50,0.50}{##1}}}
\expandafter\def\csname PY@tok@c1\endcsname{\let\PY@it=\textit\def\PY@tc##1{\textcolor[rgb]{0.25,0.50,0.50}{##1}}}
\expandafter\def\csname PY@tok@cs\endcsname{\let\PY@it=\textit\def\PY@tc##1{\textcolor[rgb]{0.25,0.50,0.50}{##1}}}

\def\PYZbs{\char`\\}
\def\PYZus{\char`\_}
\def\PYZob{\char`\{}
\def\PYZcb{\char`\}}
\def\PYZca{\char`\^}
\def\PYZam{\char`\&}
\def\PYZlt{\char`\<}
\def\PYZgt{\char`\>}
\def\PYZsh{\char`\#}
\def\PYZpc{\char`\%}
\def\PYZdl{\char`\$}
\def\PYZhy{\char`\-}
\def\PYZsq{\char`\'}
\def\PYZdq{\char`\"}
\def\PYZti{\char`\~}
% for compatibility with earlier versions
\def\PYZat{@}
\def\PYZlb{[}
\def\PYZrb{]}
\makeatother


    % For linebreaks inside Verbatim environment from package fancyvrb. 
    \makeatletter
        \newbox\Wrappedcontinuationbox 
        \newbox\Wrappedvisiblespacebox 
        \newcommand*\Wrappedvisiblespace {\textcolor{red}{\textvisiblespace}} 
        \newcommand*\Wrappedcontinuationsymbol {\textcolor{red}{\llap{\tiny$\m@th\hookrightarrow$}}} 
        \newcommand*\Wrappedcontinuationindent {3ex } 
        \newcommand*\Wrappedafterbreak {\kern\Wrappedcontinuationindent\copy\Wrappedcontinuationbox} 
        % Take advantage of the already applied Pygments mark-up to insert 
        % potential linebreaks for TeX processing. 
        %        {, <, #, %, $, ' and ": go to next line. 
        %        _, }, ^, &, >, - and ~: stay at end of broken line. 
        % Use of \textquotesingle for straight quote. 
        \newcommand*\Wrappedbreaksatspecials {% 
            \def\PYGZus{\discretionary{\char`\_}{\Wrappedafterbreak}{\char`\_}}% 
            \def\PYGZob{\discretionary{}{\Wrappedafterbreak\char`\{}{\char`\{}}% 
            \def\PYGZcb{\discretionary{\char`\}}{\Wrappedafterbreak}{\char`\}}}% 
            \def\PYGZca{\discretionary{\char`\^}{\Wrappedafterbreak}{\char`\^}}% 
            \def\PYGZam{\discretionary{\char`\&}{\Wrappedafterbreak}{\char`\&}}% 
            \def\PYGZlt{\discretionary{}{\Wrappedafterbreak\char`\<}{\char`\<}}% 
            \def\PYGZgt{\discretionary{\char`\>}{\Wrappedafterbreak}{\char`\>}}% 
            \def\PYGZsh{\discretionary{}{\Wrappedafterbreak\char`\#}{\char`\#}}% 
            \def\PYGZpc{\discretionary{}{\Wrappedafterbreak\char`\%}{\char`\%}}% 
            \def\PYGZdl{\discretionary{}{\Wrappedafterbreak\char`\$}{\char`\$}}% 
            \def\PYGZhy{\discretionary{\char`\-}{\Wrappedafterbreak}{\char`\-}}% 
            \def\PYGZsq{\discretionary{}{\Wrappedafterbreak\textquotesingle}{\textquotesingle}}% 
            \def\PYGZdq{\discretionary{}{\Wrappedafterbreak\char`\"}{\char`\"}}% 
            \def\PYGZti{\discretionary{\char`\~}{\Wrappedafterbreak}{\char`\~}}% 
        } 
        % Some characters . , ; ? ! / are not pygmentized. 
        % This macro makes them "active" and they will insert potential linebreaks 
        \newcommand*\Wrappedbreaksatpunct {% 
            \lccode`\~`\.\lowercase{\def~}{\discretionary{\hbox{\char`\.}}{\Wrappedafterbreak}{\hbox{\char`\.}}}% 
            \lccode`\~`\,\lowercase{\def~}{\discretionary{\hbox{\char`\,}}{\Wrappedafterbreak}{\hbox{\char`\,}}}% 
            \lccode`\~`\;\lowercase{\def~}{\discretionary{\hbox{\char`\;}}{\Wrappedafterbreak}{\hbox{\char`\;}}}% 
            \lccode`\~`\:\lowercase{\def~}{\discretionary{\hbox{\char`\:}}{\Wrappedafterbreak}{\hbox{\char`\:}}}% 
            \lccode`\~`\?\lowercase{\def~}{\discretionary{\hbox{\char`\?}}{\Wrappedafterbreak}{\hbox{\char`\?}}}% 
            \lccode`\~`\!\lowercase{\def~}{\discretionary{\hbox{\char`\!}}{\Wrappedafterbreak}{\hbox{\char`\!}}}% 
            \lccode`\~`\/\lowercase{\def~}{\discretionary{\hbox{\char`\/}}{\Wrappedafterbreak}{\hbox{\char`\/}}}% 
            \catcode`\.\active
            \catcode`\,\active 
            \catcode`\;\active
            \catcode`\:\active
            \catcode`\?\active
            \catcode`\!\active
            \catcode`\/\active 
            \lccode`\~`\~ 	
        }
    \makeatother

    \let\OriginalVerbatim=\Verbatim
    \makeatletter
    \renewcommand{\Verbatim}[1][1]{%
        %\parskip\z@skip
        \sbox\Wrappedcontinuationbox {\Wrappedcontinuationsymbol}%
        \sbox\Wrappedvisiblespacebox {\FV@SetupFont\Wrappedvisiblespace}%
        \def\FancyVerbFormatLine ##1{\hsize\linewidth
            \vtop{\raggedright\hyphenpenalty\z@\exhyphenpenalty\z@
                \doublehyphendemerits\z@\finalhyphendemerits\z@
                \strut ##1\strut}%
        }%
        % If the linebreak is at a space, the latter will be displayed as visible
        % space at end of first line, and a continuation symbol starts next line.
        % Stretch/shrink are however usually zero for typewriter font.
        \def\FV@Space {%
            \nobreak\hskip\z@ plus\fontdimen3\font minus\fontdimen4\font
            \discretionary{\copy\Wrappedvisiblespacebox}{\Wrappedafterbreak}
            {\kern\fontdimen2\font}%
        }%
        
        % Allow breaks at special characters using \PYG... macros.
        \Wrappedbreaksatspecials
        % Breaks at punctuation characters . , ; ? ! and / need catcode=\active 	
        \OriginalVerbatim[#1,codes*=\Wrappedbreaksatpunct]%
    }
    \makeatother

    % Exact colors from NB
    \definecolor{incolor}{HTML}{303F9F}
    \definecolor{outcolor}{HTML}{D84315}
    \definecolor{cellborder}{HTML}{CFCFCF}
    \definecolor{cellbackground}{HTML}{F7F7F7}
    
    % prompt
    \makeatletter
    \newcommand{\boxspacing}{\kern\kvtcb@left@rule\kern\kvtcb@boxsep}
    \makeatother
    \newcommand{\prompt}[4]{
        \ttfamily\llap{{\color{#2}[#3]:\hspace{3pt}#4}}\vspace{-\baselineskip}
    }
    

    
    % Prevent overflowing lines due to hard-to-break entities
    \sloppy 
    % Setup hyperref package
    \hypersetup{
      breaklinks=true,  % so long urls are correctly broken across lines
      colorlinks=true,
      urlcolor=urlcolor,
      linkcolor=linkcolor,
      citecolor=citecolor,
      }
    % Slightly bigger margins than the latex defaults
    
    \geometry{verbose,tmargin=1in,bmargin=1in,lmargin=1in,rmargin=1in}
    
    

\begin{document}
    
    
    
\maketitle

\newpage

\tableofcontents

\newpage

\listoffigures

\newpage

\listoftables

\newpage

    
    

    
    \hypertarget{initialisation}{%
\section{Initialisation}\label{initialisation}}

    \hypertarget{import-packages}{%
\subsection{import packages}\label{import-packages}}

    \begin{Verbatim}[commandchars=\\\{\}]
The required packages for this Notebook are:

    \end{Verbatim}

    
        
    \begin{table}[ht] \rowcolors{2}{white}{gray!25}
\begin{tabular}[l]{ll}
\toprule
Package & Version\\ 
\midrule
ipypublish & 0.10.10\\ 
prettytable & 0.7.2\\ 
numpy & 1.16.5\\ 
pandas & 0.25.1\\ 
matplotlib & 3.1.1\\ 
ipython & 7.12.0\\ 
\bottomrule 
 \end{tabular}
\end{table}

    
    

    \hypertarget{definition-of-functions}{%
\subsection{definition of functions}\label{definition-of-functions}}

    \hypertarget{function-tp-display-beatiful-tables}{%
\subsubsection{function tp display beatiful
tables}\label{function-tp-display-beatiful-tables}}

    \begin{verbatim}
def format_for_print(df,n=0,wide=False):
    df2=df.round(n).reset_index()
    col=[w.replace("_", " ") for w in list(df2.columns)]
    return pt.PrettyTable(df2.values,col,wide_table=wide)
\end{verbatim}

    

    \hypertarget{import-data}{%
\section{import data}\label{import-data}}

    The data for this example are generated with the demo version of
meteonorm 7

the data will be dealing with the weather data of the berlin tempelhof
weatherstation on monthly basis.

    
        
    \begin{table}[ht] \rowcolors{2}{white}{gray!25}
\begin{tabular}[l]{lll}
\toprule
Symbol & Unit & Description\\ 
\midrule
G Gh & kWh/m² & Global solar irradiance monthly averages\\ 
G Dh & kWh/m² & Diffuse solar irradiance monthly averages\\ 
Ta & °C & Air temperature\\ 
Td & °C & Dew point\\ 
FF & m/s & wind speed\\ 
\bottomrule 
 \end{tabular}
\end{table}

    
    

    
        
    
    \begin{verbatim}
(['Symbol', 'Unit', 'Description'],
 [['G Gh', 'kWh/m²', 'Global solar irradiance monthly averages'],
  ['G Dh', 'kWh/m²', 'Diffuse solar irradiance monthly averages'],
  ['Ta', '°C', 'Air temperature'],
  ['Td', '°C', 'Dew point'],
  ['FF', 'm/s', 'wind speed']])
    \end{verbatim}

    
    

    \hypertarget{text-and-images}{%
\section{Text and Images}\label{text-and-images}}

    \hypertarget{markdown-image-and-text}{%
\subsection{markdown image and text}\label{markdown-image-and-text}}

    
        
    \begin{figure}
        \begin{center}\adjustimage{max size={0.9\linewidth}{0.4\paperheight}}{my_notebook_files/my_notebook_16_0.png}\end{center}
        \caption{}
        \label{}
    \end{figure}
    
    

    here above image is a markdown image and the present text is a markdown
text

    \hypertarget{code-images-and-text}{%
\subsection{Code images and text}\label{code-images-and-text}}

    
        
    \begin{figure}
        \begin{center}\adjustimage{max size={0.9\linewidth}{0.4\paperheight}}{my_notebook_files/my_notebook_19_0.png}\end{center}
        \caption{}
        \label{}
    \end{figure}
    
    

    \begin{Verbatim}[commandchars=\\\{\}]
 The here above images are generated from the concatenation of 2 images using
nb\_setup.images\_hconcat and this text was written using the print statement
    \end{Verbatim}

    \hypertarget{example-of-a-table}{%
\section{Example of a table}\label{example-of-a-table}}

    \hypertarget{table-small}{%
\subsection{table small}\label{table-small}}

    \begin{Verbatim}[commandchars=\\\{\}]
Monthly weather data from Berlin
    \end{Verbatim}

    \begin{Verbatim}[commandchars=\\\{\}]
units
    \end{Verbatim}

    
        
    \begin{table}[ht] \rowcolors{2}{white}{gray!25}
\begin{tabular}[l]{lllllll}
\toprule
index & month & G Gh & G Dh & Ta & Td & FF\\ 
\midrule
0 & nan & kWh/m² & kWh/m² & °C & °C & m/s\\ 
\bottomrule 
 \end{tabular}
\end{table}

    
    

    \begin{Verbatim}[commandchars=\\\{\}]
data
    \end{Verbatim}

    
        
    \begin{table}[ht] \rowcolors{2}{white}{gray!25}
\begin{tabular}[l]{llllll}
\toprule
index & G Gh & G Dh & Ta & Td & FF\\ 
\midrule
1.0 & 20.0 & 12.0 & 1.0 & -2.0 & 4.0\\ 
2.0 & 36.0 & 20.0 & 2.0 & -1.0 & 4.0\\ 
3.0 & 76.0 & 43.0 & 4.0 & 0.0 & 5.0\\ 
4.0 & 124.0 & 67.0 & 10.0 & 3.0 & 3.0\\ 
5.0 & 154.0 & 78.0 & 15.0 & 8.0 & 5.0\\ 
6.0 & 164.0 & 85.0 & 17.0 & 11.0 & 4.0\\ 
7.0 & 160.0 & 81.0 & 19.0 & 13.0 & 4.0\\ 
8.0 & 136.0 & 68.0 & 19.0 & 13.0 & 4.0\\ 
9.0 & 94.0 & 47.0 & 14.0 & 10.0 & 4.0\\ 
10.0 & 55.0 & 32.0 & 10.0 & 7.0 & 4.0\\ 
11.0 & 24.0 & 16.0 & 5.0 & 3.0 & 4.0\\ 
12.0 & 15.0 & 10.0 & 1.0 & -1.0 & 6.0\\ 
\bottomrule 
 \end{tabular}
\end{table}

    
    

    \hypertarget{table-wide}{%
\subsection{Table wide}\label{table-wide}}

    This is an example of wide table with random float rounded to 3 position
after comma

    
        
    \begin{table}[ht] \rowcolors{2}{white}{gray!25}
\resizebox{\textwidth}{!}{%
\begin{tabular}[l]{llllllllllllllll}
\toprule
index & col 1 & col 2 & col 3 & col 4 & col 5 & col 6 & col 7 & col 8 & col 9 & col 10 & col 11 & col 12 & col 13 & col 14 & col 15\\ 
\midrule
0.0 & 203.296 & 170.353 & 120.157 & 69.562 & 175.545 & 83.606 & 42.985 & 15.006 & 201.066 & 244.873 & 205.922 & 236.361 & 23.049 & 157.099 & 103.137\\ 
1.0 & 107.713 & 54.042 & 86.814 & 24.314 & 24.726 & 110.548 & 77.4 & 229.568 & 17.266 & 244.378 & 90.072 & 20.769 & 114.308 & 214.715 & 146.588\\ 
2.0 & 239.27 & 212.6 & 183.042 & 239.396 & 211.63 & 208.054 & 41.616 & 122.705 & 82.545 & 87.526 & 67.349 & 160.098 & 85.924 & 20.963 & 182.954\\ 
3.0 & 122.43 & 28.443 & 207.225 & 166.124 & 124.056 & 99.523 & 60.607 & 113.076 & 178.929 & 139.558 & 110.456 & 182.563 & 139.969 & 239.076 & 210.876\\ 
4.0 & 164.397 & 3.281 & 208.881 & 101.49 & 185.898 & 136.717 & 181.302 & 70.342 & 59.093 & 101.207 & 95.574 & 17.56 & 18.208 & 37.493 & 186.145\\ 
5.0 & 236.33 & 232.673 & 42.856 & 169.306 & 16.767 & 147.67 & 146.935 & 194.675 & 11.01 & 195.95 & 141.447 & 27.843 & 179.796 & 7.291 & 45.787\\ 
6.0 & 56.715 & 189.396 & 184.271 & 59.158 & 149.959 & 245.124 & 111.463 & 155.38 & 134.087 & 3.994 & 32.424 & 248.415 & 153.74 & 205.65 & 219.714\\ 
7.0 & 191.875 & 82.788 & 240.251 & 143.926 & 129.835 & 133.587 & 100.97 & 241.174 & 168.388 & 25.068 & 97.679 & 241.481 & 171.997 & 190.412 & 198.979\\ 
8.0 & 135.594 & 242.375 & 128.774 & 245.487 & 140.37 & 163.317 & 223.338 & 61.157 & 197.576 & 162.683 & 60.22 & 41.592 & 103.603 & 53.371 & 93.509\\ 
9.0 & 147.287 & 134.721 & 108.253 & 35.658 & 71.804 & 1.591 & 221.89 & 31.861 & 86.124 & 103.211 & 160.752 & 59.993 & 185.295 & 252.617 & 239.529\\ 
10.0 & 107.99 & 242.826 & 20.87 & 251.699 & 239.98 & 53.651 & 63.281 & 96.372 & 25.601 & 222.378 & 127.271 & 236.955 & 195.604 & 201.073 & 50.68\\ 
11.0 & 66.843 & 193.688 & 77.929 & 156.808 & 116.903 & 215.925 & 47.699 & 67.468 & 119.391 & 230.679 & 19.317 & 104.118 & 143.069 & 177.011 & 96.542\\ 
12.0 & 103.571 & 51.331 & 96.66 & 90.633 & 241.807 & 47.074 & 190.508 & 219.786 & 228.043 & 96.151 & 187.646 & 233.202 & 19.263 & 227.613 & 215.424\\ 
13.0 & 101.645 & 31.607 & 51.963 & 92.539 & 238.774 & 70.382 & 209.095 & 91.714 & 49.849 & 100.635 & 177.559 & 61.67 & 196.796 & 77.412 & 34.366\\ 
14.0 & 30.906 & 154.445 & 20.109 & 92.019 & 72.787 & 6.324 & 245.944 & 156.275 & 198.169 & 52.48 & 228.02 & 214.756 & 106.027 & 113.523 & 215.148\\ 
15.0 & 250.627 & 51.754 & 53.844 & 76.019 & 185.036 & 216.172 & 58.179 & 210.186 & 222.274 & 175.591 & 153.068 & 30.0 & 70.486 & 35.551 & 47.623\\ 
16.0 & 213.321 & 104.582 & 18.562 & 67.508 & 200.546 & 143.43 & 201.902 & 184.092 & 76.982 & 135.874 & 224.323 & 63.053 & 174.118 & 141.251 & 206.455\\ 
17.0 & 29.389 & 93.519 & 211.335 & 60.47 & 30.274 & 212.46 & 196.486 & 173.071 & 77.239 & 98.726 & 4.193 & 198.201 & 208.69 & 228.378 & 49.473\\ 
18.0 & 195.497 & 101.95 & 249.769 & 200.754 & 83.965 & 239.988 & 39.146 & 16.22 & 32.833 & 2.431 & 231.399 & 234.421 & 192.944 & 166.454 & 143.415\\ 
19.0 & 158.16 & 170.841 & 119.892 & 33.599 & 24.321 & 91.204 & 64.33 & 121.132 & 98.159 & 161.71 & 206.886 & 142.903 & 246.62 & 11.874 & 252.001\\ 
\bottomrule 
 \end{tabular}
}\end{table}

    
    

    \hypertarget{example-of-chart}{%
\section{Example of chart}\label{example-of-chart}}

    \hypertarget{chart-bar}{%
\subsection{chart bar}\label{chart-bar}}

        
    \begin{figure}
        \begin{center}\adjustimage{max size={0.9\linewidth}{0.4\paperheight}}{my_notebook_files/my_notebook_31_0.pdf}\end{center}
        \caption{}
        \label{}
    \end{figure}
    
    
    \hypertarget{chart-scatter}{%
\subsection{chart scatter}\label{chart-scatter}}

        
    \begin{figure}
        \begin{center}\adjustimage{max size={0.9\linewidth}{0.4\paperheight}}{my_notebook_files/my_notebook_33_0.pdf}\end{center}
        \caption{}
        \label{}
    \end{figure}
    
    
    \hypertarget{chart-line}{%
\subsection{chart line}\label{chart-line}}

        
    \begin{figure}
        \begin{center}\adjustimage{max size={0.9\linewidth}{0.4\paperheight}}{my_notebook_files/my_notebook_35_0.pdf}\end{center}
        \caption{}
        \label{}
    \end{figure}
    
    
    \hypertarget{example-of-matematic-formulas}{%
\section{Example of matematic
formulas}\label{example-of-matematic-formulas}}

    example 1:

\[ y = a x + b \]

with:

\begin{itemize}
\tightlist
\item
  y : Ordonate
\item
  x : Abscissa
\item
  a : Slope
\item
  b : Initial value
\end{itemize}

    example 2:

\[\sum_{i=1}^{\infty} i = \frac{n(n+1)}{2}\]

    example 3:

\[ \int\limits_0^\infty {\sqrt{x} e^{ - x} dx} = \frac{1}{2}\sqrt{\pi}\]

    \hypertarget{generation-of-the-template}{%
\section{Generation of the template}\label{generation-of-the-template}}

    \begin{Verbatim}[commandchars=\\\{\}]
Overwriting my\_template.tplx
    \end{Verbatim}


    % Add a bibliography block to the postdoc
    
    
    
\end{document}
